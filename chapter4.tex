\chapter{DESCRIPTION OF TOOLS AND TECHNOLOGIES}

\section{JAVA SWINGS AND APPLETS}
A Java applet is a small application that is written in the Java programming language, or another programming language that compiles to Java bytecode, and delivered to users in the form of Java bytecode. The user launches the Java applet from a web page, and the applet is then executed within a Java virtual machine (JVM) in a process separate from the web browser itself. A Java applet can appear in a frame of the web page, a new application window, Sun's AppletViewer, or a stand-alone tool for testing applets. Java applets were introduced in the first version of the Java language, which was released in 1995.Java applets are usually written in Java, but other languages such as Jython,  JRuby, Pascal, Scala, or Eiffel (via SmartEiffel)may be used as well.

Java applets run at very fast speeds and, until 2011, they were many times faster than JavaScript.[11] Unlike JavaScript, Java applets had access to 3D hardware acceleration, making them well-suited for non-trivial, computation-intensive visualizations. As browsers have gained support for hardware-accelerated graphics thanks to the canvas technology (or specifically WebGL in the case of 3D graphics), as well as just-in-time compiled JavaScript, the speed difference has become less noticeable.Since Java bytecode is cross-platform (or platform independent), Java applets can be executed by browsers (or other clients) for many platforms, including Microsoft Windows, FreeBSD, Unix, macOS and Linux.Java applet technology has been marked for deprecation

The Applets are used to provide interactive features to web applications that cannot be provided by HTML alone.They can capture mouse input and also have controls like buttons or check boxes. In response to user actions, an applet can change the provided graphic content. This makes applets well-suited for demonstration, visualization, and teaching. There are online applet collections for studying various subjects, from physics to heart physiology.An applet can also be a text area only; providing, for instance, a cross-platform command-line interface to some remote system. If needed, an applet can leave the dedicated area and run as a separate window. However, applets have very little control over web page content outside the applet's dedicated area, so they are less useful for improving the site appearance in general, unlike other types of browser extensions Applets can also play media in formats that are not natively supported by the browser. 

Pages coded in HTML may embed parameters within them that are passed to the applet. Because of this, the same applet may have a different appearance depending on the parameters that were passed.

As applets were available before CSS and DHTML were standard, they were also widely used for trivial effects such as rollover navigation buttons. Heavily criticized, this usage is now declining.

Java applets are executed in a sandbox by most web browsers, preventing them from accessing local data like the clipboard or file system. The code of the applet is downloaded from a web server, after which the browser either embeds the applet into a web page or opens a new window showing the applet's user interface.

A Java applet extends the class java.applet.Applet, or in the case of a Swing applet, javax.swing.JApplet. The class which must override methods from the applet class to set up a user interface inside itself (Applet) is a descendant of Panel which is a descendant of Container. As applet inherits from container, it has largely the same user interface possibilities as an ordinary Java application, including regions with user specific visualization.

The first implementations involved downloading an applet class by class. While classes are small files, there are often many of them, so applets got a reputation as slow-loading components. However, since .jars were introduced, an applet is usually delivered as a single file that has a size similar to an image file (hundreds of kilobytes to several megabytes).The domain from where the applet executable has been downloaded is the only domain to which the usual (unsigned) applet is allowed to communicate. This domain can be different from the domain where the surrounding HTML document is hosted.Java system libraries and runtimes are backwards-compatible, allowing one to write code that runs both on current and on future versions of the Java virtual machine.

Swing is a GUI widget toolkit for Java. It is part of Oracle's Java Foundation Classes (JFC) – an API for providing a graphical user interface (GUI) for Java programs.

Swing was developed to provide a more sophisticated set of GUI components than the earlier Abstract Window Toolkit (AWT). Swing provides a native look and feel that emulates the look and feel of several platforms, and also supports a pluggable look and feel that allows applications to have a look and feel unrelated to the underlying platform. It has more powerful and flexible components than AWT. In addition to familiar components such as buttons, check boxes and labels, Swing provides several advanced components such as tabbed panel, scroll panes, trees, tables, and lists.

Unlike AWT components, Swing components are not implemented by platform-specific code. Instead, they are written entirely in Java and therefore are platform-independent. The term "lightweight" is used to describe such an element.

Swing is a highly modular-based architecture, which allows for the "plugging" of various custom implementations of specified framework interfaces: Users can provide their own custom implementation(s) of these components to override the default implementations using Java's inheritance mechanism.

Swing is a component-based framework, whose components are all ultimately derived from the javax.swing.JComponent class. Swing objects asynchronously fire events, have bound properties, and respond to a documented set of methods specific to the component. Swing components are Java Beans components, compliant with the Java Beans Component Architecture specifications.
Since early versions of Java, a portion of the Abstract Window Toolkit (AWT) has provided platform-independent APIs for user interface components. In AWT, each component is rendered and controlled by a native peer component specific to the underlying windowing system.

By contrast, Swing components are often described as lightweight because they do not require allocation of native resources in the operating system's windowing toolkit. The AWT components are referred to as heavyweight components. 

Much of the Swing API is generally a complementary extension of the AWT rather than a direct replacement. In fact, every Swing lightweight interface ultimately exists within an AWT heavyweight component because all of the top-level components in Swing (JApplet, JDialog, JFrame, and JWindow) extend an AWT top-level container.

Swing's high level of flexibility is reflected in its inherent ability to override the native host operating system (OS)'s GUI controls for displaying itself. Swing "paints" its controls using the Java 2D APIs, rather than calling a native user interface toolkit. Thus, a Swing component does not have a corresponding native OS GUI component, and is free to render itself in any way that is possible with the underlying graphics GUIs.

However, at its core, every Swing component relies on an AWT container, since (Swing's) JComponent extends (AWT's) Container. This allows Swing to plug into the host OS's GUI management framework, including the crucial device/screen mappings and user interactions, such as key presses or mouse movements. Swing simply "transposes" its own (OS-agnostic) semantics over the underlying (OS-specific) components. So, for example, every Swing component paints its rendition on the graphic device in response to a call to component.paint(), which is defined in (AWT) Container. But unlike AWT components, which delegated the painting to their OS-native "heavyweight" widget, Swing components are responsible for their own rendering.

This transposition and decoupling is not merely visual, and extends to Swing's management and application of its own OS-independent semantics for events fired within its component containment hierarchies. Generally speaking, the Swing architecture delegates the task of mapping the various flavors of OS GUI semantics onto a simple, but generalized, pattern to the AWT container. Building on that generalized platform, it establishes its own rich and complex GUI semantics in the form of the JComponent model.

\section{CSS}
(if required)
\section{JAVA}
Java is a general-purpose computer programming language that is concurrent, class-based, object-oriented,and specifically designed to have as few implementation dependencies as possible. It is intended to let application developers "write once, run anywhere" (WORA), meaning that compiled Java code can run on all platforms that support Java without the need for recompilation. Java applications are typically compiled to bytecode that can run on any Java virtual machine (JVM) regardless of computer architecture. As of 2016, Java is one of the most popular programming languages in use,particularly for client-server web applications, with a reported 9 million developers. Java was originally developed by James Gosling at Sun Microsystems (which has since been acquired by Oracle Corporation) and released in 1995 as a core component of Sun Microsystems' Java platform. The language derives much of its syntax from C and C++, but it has fewer low-level facilities than either of them.
One design goal of Java is portability, which means that programs written for the Java platform must run similarly on any combination of hardware and operating system with adequate runtime support. This is achieved by compiling the Java language code to an intermediate representation called Java bytecode, instead of directly to architecture-specific machine code. Java bytecode instructions are analogous to machine code, but they are intended to be executed by a virtual machine (VM) written specifically for the host hardware. End users commonly use a Java Runtime Environment (JRE) installed on their own machine for standalone Java applications, or in a web browser for Java applets.

\section{MYSQL}
MySQL is a Relational Database Management System (RDBMS). MySQL server
can manage many databases at the same time. In fact, many people might have different
databases managed by a single MySQL server. Each database consists of a structure to hold
the data and the data itself. A data-base can exist without data, only a structure, be totally
empty, twiddling its thumbs and waiting for data to be stored in it.

Data in a database is stored in one or more tables. You must create the data-base and
the tables before you can add any data to the database. First you create the empty database.

Then you add empty tables to the database. Database tables are organized like other tables
that you’re used in rows and columns. Each row represents an entity in the database, such as
a customer, a book, or a project. Each column contains an item of information about the
entity, such as a customer name, a book name, or a project start date. The place where a
particular row and column intersect, the individual cell of the table, is called a field. Tables
in databases can be related. Often a row in one table is related to several rows in another
table. For instance, you might have a database containing data about books you own. You
would have a book table and an author table. One row in the author table might contain
information about the author of several books in the book table. When tables are related.
you include a column in one table to hold data that matches data in the column of another
table.

MySQL, the most popular Open Source SQL database management system, is
developed, distributed, and supported by MySQL AB. MySQL AB is a commercial company,
founded by the MySQL developers. It is a second generation Open Source company that
unites Open Source values and methodology with a successful business model.
\begin{itemize}
\item{MySQL is a database management system.
A database is a structured collection of data. It may be anything from a simple
shopping list to a picture gallery or the vast amounts of information in a corporate
network. To add, access, and process data stored in a computer database, you need a
database management system such as MySQL Server. Since computers are very good
at handling large amounts of data, database management systems play a central role in
computing, as standalone utilities, or as parts of other applications.}
\item{MySQL is a relational database management system.
A relational database stores data in separate tables rather than putting all the data in
one big storeroom. This adds speed and flexibility. The SQL part of “MySQL” stands
for “Structured Query Language.” SQL is the most common standardized language
used to access databases and is defined by the ANSI/ISO SQL Standard. The SQL
standard has been evolving since 1986 and several versions exist. “SQL-92” refers to
the standard released in 1992, “SQL:1999” refers to the standard released in 1999, and
“SQL:2003” refers to the current version of the standard. We use the phrase “the SQL
standard” to mean the current version of the SQL Standard at any time.}
\item{MySQL software is Open Source.
Open Source means that it is possible for anyone to use and modify the software.
Anybody can download the MySQL software from the Internet and use it without
paying anything. If you wish, you may study the source code and change it to suit
your needs. The MySQL software uses the GPL (GNU General Public License), to
define what you may and may not do with the software in different situations. The
MySQL Database Server is very fast, reliable, and easy to use.

MySQL Server was originally developed to handle large databases much faster than
existing solutions and has been successfully used in highly demanding production
environments for several years. Although under constant development, MySQL
Server today offers a rich and useful set of functions. Its connectivity, speed, and
security make MySQL Server highly suited for accessing databases on the Internet.}
\item{MySQL Server works in client/server or embedded systems.
The MySQL Database Software is a client/server system that consists of a multi-
threaded SQL server that supports different back ends, several different client
programs and libraries, administrative tools, and a wide range of application
programming interfaces (APIs).}
\end{itemize}


