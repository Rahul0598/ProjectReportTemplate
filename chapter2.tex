\chapter{Requirement Analysis}

\section{Hardware Requirements}
The Hardware requirements are very minimal and the program can be run on most of
the machines. \\
Processor : Pentium4 processor\\
Processor Speed : 2.4 GHz\\
RAM : 1 GB\\
Storage Space : 40 GB\\
Monitor Resolution : 1024*768 or 1336*768 or 1280*1024\\
\thispagestyle{fancy}

\section{Software Requirements}
Operating System - Windows XP\\
IDE - MYSQL Workbench,Eclipse\\
Additional tool - NetBeans\\
\thispagestyle{fancy}

\section{Functional Requirements}
\subsection{Major Entities}
\textbf{ACCOUNT}: Account in which the detail of account holders will be stored.\\
\textbf{DEPOSIT}: We can store deposit details in the deposit entity or table.\\
\textbf{New\_pin}: We can change user pin and the new pin will updated in change\_pin entity.\\
\textbf{TRANSFER}: Customer can find their transfer detail in the transfer table or entity.\\
\textbf{WITHDRAW}: Customer can withdraw money and can find the detail of withdraw in withdraw table.\\
\textbf{PIN\_UPDATE\_RECORD}: Customer can know about the new\_pin updation by
pin\_update\_record entity.\\
\thispagestyle{fancy}

\subsection{End User Requirements}
\begin{enumerate}
\item Main Goals:
	\begin{itemize}
	\item Our motto is to develop a software program for managing the entire bank process
related to Administration accounts customer accounts and to keep each every track about
their property and their various transaction processes efficiently.
	\item Hereby, our main objective is the customers satisfaction considering todays faster in
the world
	\end{itemize}
\item Customer Satisfaction:
	\begin{itemize}
	\item Client can do his operations comfortably without any risk or losing of his privacy.
	\item Our software will perform and fulfill all the tasks that any customer would desire.
	\end{itemize}		
\item Saving Customer Time:
	\begin{itemize}
	\item Client doesn't need to go to the bank to do small operation.	\end{itemize}
\end{enumerate}
\thispagestyle{fancy}

\subsection{Java Swings and Applets}
A Java applet is a small application that is written in the Java programming
language, or another programming language that compiles to Java byte code, and delivered to
users in the form of Java byte code. The user launches the Java applet from a web page, and
the applet is then executed within a Java virtual machine (JVM) in a process separate from
the web browser itself. A Java applet can appear in a frame of the web page, a new
application window, Sun's AppletViewer, or a stand-alone tool for testing applets. Java
applets were introduced in the first version of the Java language, which was released in
1995.Java applets are usually written in Java, but other languages such as Jython, JRuby,
Pascal, Scala, or Eiffel (via SmartEiffel)may be used as well.\\
Java applets run at very fast speeds and, until 2011, they were many times faster than
JavaScript. [11] Unlike JavaScript, Java applets had access to 3D hardware acceleration,
making them well-suited for non-trivial, computation-intensive visualizations. As browsers
have gained support for hardware-accelerated graphics thanks to the canvas technology (or
specifically WebGL in the case of 3D graphics), as well as just-in-time compiled JavaScript,
the speed difference has become less noticeable. Since Java bytecode is cross-platform (or
platform independent), Java applets can be executed by browsers (or other clients) for many
platforms, including Microsoft Windows, FreeBSD, Unix, macOS and Linux. Java applet
technology has been marked for deprecation.\\
The Applets are used to provide interactive features to web applications that cannot be
provided by HTML alone. They can capture mouse input and also have controls like buttons
or check boxes. In response to user actions, an applet can change the provided graphic
content. This makes applets well-suited for demonstration, visualization, and teaching. There
are online applet collections for studying various subjects, from physics to heart physiology.
An applet can also be a text area only; providing, for instance, a cross-platform command-
line interface to some remote system. If needed, an applet can leave the dedicated area and
run as a separate window. However, applets have very little control over web page content
outside the applet's dedicated area, so they are less useful for improving the site appearance in general, unlike other types of browser extensions Applets can also play media in formats that are not natively supported by the browser.\\

\thispagestyle{fancy}